\documentclass{article}
\usepackage[utf8]{inputenc}

\usepackage[T2A]{fontenc}
\usepackage[utf8]{inputenc}
\usepackage[russian]{babel}

\usepackage{amsmath}

\title{Вычислительная техника}
\author{Лисид Лаконский}
\date{October 2022}

\begin{document}

\maketitle
\tableofcontents
\pagebreak

\section{Вычислительная техника - 03.10.2022}

Ревью прошлого зантия: совершенная дизъюнктивная (или) форма, соершенная конъюнктивная (и) нормальная форма, минимазция с помощью трех методов: алгебраического, с помощью карт Карно, с помощью диаграммы Вейча.

\subsection{Табличные методы минимизации}

\subsubsection{Минимизация с помощью карт Карно}

$
\begin{pmatrix}
     & 0 & 1 \\
    0 \\
    1
\end{pmatrix}
$ - шаблон карты Карно для функции, принимающей два аргумента.

$
\begin{pmatrix}
     & 00 & 01 & 11 & 10 \\
    0 \\
    1
\end{pmatrix}
$ - шаблон карты Карно для функции, принимающей три аргумента.

$
\begin{pmatrix}
     & 00 & 01 & 11 & 10 \\
    00 \\
    01 \\
    11 \\
    10
\end{pmatrix}
$ - шаблон карты Карно для функции, принимающей четыре аргумента.

Основные принципы склейки:

\begin{enumerate}
    \item Склейку клеток одной и той же карты Карно можно осуществлять как по единицам (a), так и по нулям (б). Первое необходимо для получения ДНФ, второе — для получения КНФ
    \item Склеивать можно только прямоугольные области с числом единиц (нулей), являющимся целой степенью двойки
    \item Рекомендуется выбирать максимально возможные области склейки
    \item Для карт Карно с числом переменных 3 и 4 применимо следующее правило: крайние клетки каждой горизонтали и каждой вертикали граничат между собой и могут объединяться в прямоугольники (топологически карта Карно представляет собой тор). Следствием этого правила является смежность всех четырёх угловых ячеек карты Карно для 4 переменных
\end{enumerate}

\subsubsection{Минимизация с помощью диаграмм Вейча}

Метод минимизации с помощью диаграмм Вейча основан на методе с применением карт Карно, однако элементы записываются иначе, более удобно для формирования итоговой формулы: лучше смотреть, что изменяется, а что нет.

Все записывается так же с помощью кода Грея, неизменяющиеся элементы подписываются так, чтобы образовывать единицу.

\subsection{Цифровые комбинационные устройства}

\subsubsection{Устройство равнозначности}

$y = (x_{1}x_{2}) + (\overline{x_{1}}\overline{x_{2}}) = \overline{\overline{x_{1}x_{2}} * \overline{\overline{x_1}\overline{x_2}}}$

Возвращает единицу, если оба аргумента равны, иначе ноль.

\subsubsection{Устройство неравнозначности}

$y = x_1\overline{x_2} + \overline{x_1}x_2$

Возвращает единицу, если оба аргумента не равны, иначе ноль.

\subsubsection{Полусумматор}

$S = x_1 \oplus x_2, P = x_{1}x_{2}$

$S$ - сумма, $P$ - перенос

\subsubsection{Комбинационный сумматор}

Комбинационный сумматор, удивительно, получается при помощии комбинации полусумматоров или других сумматоров.

Схемы тут не будет, так как в LaTeX крайне неудобно прикреплять картинки. По крайней мере, мне лень сейчас разбираться, как тут в Overleaf это делать.

Складываются аргументы, а потом результат работы сумматора складывается с переносом.

\end{document}
